\documentclass{article}
\usepackage{graphicx}
\usepackage[utf8]{inputenc}
\usepackage{xeCJK}
\usepackage{amsmath}
\usepackage{bm}
\usepackage{minted}
\usepackage{hyperref}
\usepackage[dvipsnames]{xcolor}
\usepackage{parskip}
\usepackage{soul}
\usepackage{cancel}
\usepackage{ragged2e}
\usepackage[normalem]{ulem}

\hypersetup{hidelinks}

\title{约瑟夫环\footnote{\url{https://leetcode-cn.com/problems/yuan-quan-zhong-zui-hou-sheng-xia-de-shu-zi-lcof/}}\footnote{更多内容请访问:\url{https://github.com/SamZhangQingChuan/Editorials}}}
\author{张晴川\\\href{mailto:qzha536@aucklanduni.ac.nz}{\texttt{qzha536@aucklanduni.ac.nz}}}

\begin{document}
\maketitle
\section{题意}
$0,1,\ldots,n-1$ 这 $n$ 个数字排成一个圆圈,从数字$0$开始,每次从这个圆圈里删除第 $m$ 个数字。求出这个圆圈里剩下的最后一个数字。

例如$0, 1, 2, 3, 4$这五个数字组成一个圆圈,从数字$0$开始每次删除第$3$个数字,则删除的前$4$个数字依次是$2;0;4;1$,因此最后剩下的数字是$3$。


\section{数据范围}
\begin{itemize}
\item $1 \le n \le 10^5$
\item $1 \le m \le 10^6$
\end{itemize}


\section{题解}

一句话题解:通过剩余 $x$ 个人时的答案反推 $x+1$ 个人时的答案。由于只剩一个人的时候答案必然为 $0$,层层反推即可。


首先我们用 $f(n)$ 表示剩余 $n$ 个人的时候,从\textbf{当前}起点开始走$f(n)$步会到达赢家的位置。


假设现在有$n$个人活着,那么\textbf{当前}起点是$0$。而杀完编号为$m-1$的人后,\textbf{新的}起点则是$m$。假设 $f(n-1)$ 已知,那么我们需要走 $m$ 步到新的起点,再走 $f(n-1)$ 步走到赢家位置。即一共需要走的步数为: 
$$
\underbrace{m}_{\text{新起点}} + \underbrace{f(n-1)}_{\text{额外步数}}
$$

由于是在环上走,所以步数需要对 $n$ 取模,所以最终结果是:
$$
f(n) = (m + f(n-1)) \bmod {n}
$$


\section{代码}
\inputminted[linenos,autogobble]{cpp}{Josephus.cpp}








\end{document}
