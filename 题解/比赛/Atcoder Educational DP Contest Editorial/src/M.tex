\section{Candies}
\subsection*{题意}
现在有 $n$ 个小孩和 $k$ 颗糖。第 $i$ 个小孩获得的糖数必须在 $[0,a_i]$ 之内,求一共有多少种把糖发完的方法。

\subsection*{数据范围}
\begin{itemize}
\item $1 \leq n \leq 100$
\item $0 \leq k \leq 10^5$
\item $0 \leq a_i \leq k$
\end{itemize}


\subsection*{题解}

有了之前背包的经验,显然我们可以设状态 ${\texttt{dp}[i][j]}$ 表示前 $i$ 个人一共发了 $j$ 颗糖的方法,并且转移方程就是枚举当前的人获得的糖数 $c$:
$$
{\texttt{dp}[i][j]} = \sum_{c = 0}^{a_i} {\texttt{dp}[i-1][j-c]}
$$

但可达鸭眉头一皱,发现事情并不简单。注意到状态总数为 $10^7$,而每个状态最多有$k \le 10^5$ 个转移,总复杂度高达 $10^{12}$,于是无法通过。

我们仔细观察上面的转移方程,可以发现同一行相邻的两个状态其实差别不大:
$$
{\texttt{dp}[i][j]} = {\texttt{dp}[i-1][j-c]} + \cdots + {\texttt{dp}[i-1][j]}
$$
$$
{\texttt{dp}[i][j+1]} = {\texttt{dp}[i-1][j-c+1]} + \cdots + {\texttt{dp}[i-1][j+1]}
$$

通过比对,可以看到其实只首尾两项,于是得到:
$$
{\texttt{dp}[i][j+1]} = {\texttt{dp}[i][j]} - {\texttt{dp}[i-1][j-c]} + {\texttt{dp}[i-1][j+1]}
$$
现在计算每个状态只需要 $O(1)$ 次运算,可以通过。

\subsection*{核心代码}
\inputminted[linenos,autogobble]{cpp}{./Code/M.cpp}
\emph{注:本题也可以使用前缀和加速,原理大同小异。}

\newpage
