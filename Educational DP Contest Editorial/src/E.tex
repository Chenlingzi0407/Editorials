\section{Knapsack 2}
\subsection*{题意}
同上题,仅数据范围不同:
\subsection*{数据范围}
\begin{itemize}
\item $1 \leq n \leq 100$
\item $1 \leq W \leq \bm{10^9}$
\item $1 \leq v_i \leq \bm{10^3}$
\end{itemize}

\subsection*{题解}

由于 $W$达到了$10^9$ 的量级,之前的$O(nW)$算法无法通过,但由于每样物品的价值上限仅为 $10^3$,我们可以另辟蹊径。设${\texttt{dp}[i][j]}$表示\textbf{``只考虑前$i$ 个物品的情况下,总价值是 $j$ 所需的最小容量"}。那么在计算${\texttt{dp}[i][j]}$的时候,所有情况依然可以分成两类考虑:
\begin{enumerate}
    \item \textbf{拿第 $i$ 件物品},那么别的物品的总价值需要凑出 $j - v_i$,而由于每样物品只能拿一件,所以我们只需要考虑前 $i-1$ 件物品的最优选取方式,即最终重量为 ${\texttt{dp}}[i-1][j-v_i] + w_i$。
    \item \textbf{不拿第 $i$ 件物品},那么别的物品需要凑出$j$的价值 。由于我们选择不拿第$i$件物品,现在只需要考虑前 $i-1$ 件物品的最优选取方式,即最小重量为 ${\texttt{dp}}[i-1][j]$。
\end{enumerate}
计算完所有状态的值后,只需要选取满足重量上限的最大价值即可。
\subsection*{核心代码}
\inputminted[linenos,autogobble]{cpp}{./Code/E.cpp}
\newpage